\chapter{Access control lists (ACLs)}
\label{chap:acls}

This chapter introduces access control lists and the DCP permissions model.



\section{Introduction to ACLs and permissions}
\sectionrule

An \textbf{access control list}, or \textbf{ACL}, is a list of permissions
granted to objects on other objects.  An object may be a user, group, roster, or
server.  Each object type defines its own set of \textbf{permission}s, which
specifies an operation or manipulation of some sort that can be done to the
object by another object.

For each permission applicable to an object, the object's ACL contains two
lists: \texttt{set}, and \texttt{unset}.  Any object in the \texttt{set} list
has the permission for the specified object.  Any object in the \texttt{unset}
list is explictly denied that permission for the specified object.  Objects not
appearing in either list are members of a pseudo-list, \texttt{unspecified}.
They may inherit permission allowals or rejections from a group, or by default.



\section{ACL format}
\sectionrule
\label{acl:format}

\begin{tabular}{l|l|l|l|l|l}
  \hline
  target & source & permission ( & target & source & permission \ldots ) \\
  \hline
\end{tabular}

\begin{description}
  \item[target] \hfill \\
  The ACL's target.
  
  \item[source] \hfill \\
  The object that holds the following permission.
  
  \item[permission] \hfill \\
  The permission that the preceding object holds.
\end{description}



\section{Retrieving an ACL}
\sectionrule

A client may request an ACL for an object using the \texttt{ACL-GET} command, as
specified:

\begin{tabular}{l|l|l|l}
  \hline
  Key           & Format (Req'd)& Description                           & Example \\
  \hline
  \hline
  \texttt{dest} & Object (Y)    & The object to retrieve an ACL for.    & \#Sporks \\
  \hline
\end{tabular}


The server's reply will be in the format specified in section \ref{acl:format}.



\section{Granting or removing a permission}
\sectionrule

A client may grant or remove a permission to an object for any object that it
has the authority to control permissions over.  The client must use the
\texttt{ACL-SET} command, as specified:

\begin{tabular}{l|l|l|l}
  \hline
  Key                   & Format (Req'd)& Description                                   & Example \\
  \hline
  \hline
  \texttt{dest}         & Object (Y)    & The object to set the permission for.         & \#Sporks \\
  \texttt{to}           & Object (Y)    & The object to grant/remove the permission to. & Pwyll@foxkit.us \\
  \texttt{permission}   & Permission (Y)& The permission to grant or remove.            & group.op \\
  \texttt{granted}      & Boolean (N)   & Whether to grant or remove; default: True.    & True \\
  \texttt{notes}        & String (N)    & Why the user is being granted the permission. & He's a good op. \\
  \hline
\end{tabular}
