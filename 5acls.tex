\chapter{Access control lists (ACLs)}
\label{chap:acls}

This chapter introduces access control lists and the DCP permissions model.

\newpage



\section{Introduction to ACLs and permissions}
\sectionrule

An \textbf{access control list}, or \textbf{ACL}, is a list of permissions
granted to objects on other objects.  An object may be a user, group, roster, or
server.  Each object type defines its own set of \textbf{permission}s, which
specifies an operation or manipulation of some sort that can be done to the
object by another object.

For each permission applicable to an object, the object's permission is either
\textbf{set}, and \textbf{unset}.  If the object has the permission \textbf{set}
it has the permission for the specified object.  If the object has the
permission \textbf{unset}, it is explictly denied that permission for the object
specified.  Objects that have a permission neither set nor unset are unspecified
and may inherit permission allowals or rejections by default, depending on the
permission.



\section{ACL format}
\sectionrule
\label{acl:format}

\begin{tabular}{|l|l|l|l|l|l|}
  \hline
  target & source & permission ( & target & source & permission \ldots ) \\
  \hline
\end{tabular}

\begin{description}
  \item[target] \hfill \\
  The ACL's target.
  
  \item[source] \hfill \\
  The object that holds the following permission.
  
  \item[permission] \hfill \\
  The permission that the preceding object holds.
\end{description}



\section{Retrieving an ACL}
\sectionrule

A client may request an ACL for an object using the \texttt{ACL-GET} command
with the destination set to the object to retrieve the ACL on.

The server's reply will be in the format specified in section \ref{acl:format}.



\section{Granting or removing a permission}
\sectionrule

A client may grant a permission to an object for any object that it has the
authority to control permissions over.  The client must use the \texttt{ACL-SET}
command with the destination set to the object to grant the permission on, as
specified:

\begin{tabular}{|l|l|l|l|}
  \hline
  Key                   & Format (Req'd)& Description                   & Example \\
  \hline
  \hline
  \texttt{user}         & Object (Y)    & The object to grant the       & Pwyll@foxkit.us \\
                        &               & permission to.                & \\
  \hline
  \texttt{permission}   & Permission (Y)& The permission to grant.      & group:op \\
  \hline
  \texttt{time}         & Time (N)      & How long the permission       & +604800 \\
                        &               & is granted.                   & \\
  \hline
  \texttt{notes}        & String (N)    & Why the user is being         & He's a good op for \\
                        &               & granted the permission.       & the conference this week. \\
  \hline
\end{tabular}

The number of \texttt{user}s must be at least as many as \texttt{permission}s.
If there are more \texttt{user}s specified than \texttt{permission}s, the last
\texttt{permission} specified is granted to all remaining \texttt{user}s.

A simple example, granting Pwyll@foxkit.us the \texttt{group:op} permission on
\#foxes:

\begin{tabular}{|l|l|l|l|l|l|}
  \hline
  * & \#foxes & user & Pwyll@foxkit.us & permission & group:op \\
  \hline
\end{tabular}

A more complex example, granting Victoria@foxkit.us the \texttt{group:ban}
permission, and both Pwyll@foxkit.us and Enfys@foxkit.us the \texttt{group:op}
permission on \#foxes, since there were three users but only two permissions:

\begin{tabular}{|l|l|l|l|l|l|}
  \hline
  * & \#foxes & user & Victoria@foxkit.us & user & Pwyll@foxkit.us \\
  \hline
  user & Enfys@foxkit.us & permission & group:ban & permission & group:op \\
  \hline
\end{tabular}
