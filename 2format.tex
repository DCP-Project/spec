\chapter{Message framing and formatting}
\label{chap:format}
This chapter describes the framing and formatting standard for messages
transmitted using DCP.



\section{Framing}
\sectionrule
Messages are terminated by two 'NULL' bytes, or \nul\nul.  Fields inside
messages are delimited by a single 'NULL' byte, or \nul.

A message sent over DCP must follow the following format:

\texttt{
\begin{tabular}{|l|l|l|l|l|l|}
  \hline
  length  & source  & dest  & command  & (keys/values)  & terminator \\
  \hline
\end{tabular}
}

% +---------------------------------------------------------------------+
% | length  | source  | dest  | command  | (keys/values)  | terminator  |
% +---------------------------------------------------------------------+

\begin{description}
\item[\ttfamily length] \hfill \\
A message always starts with its length, which MUST NOT include the length of
the length field and the terminating \nul.  The length MUST be sent as a big-
endian short.  The maximum length of a message is 1370 bytes.  Any DCP
implementation that receives a message with a length of more than 1370 bytes
MUST reject the message as invalid. The server SHOULD send an error message back
to the object that sent the message.  The server MUST NOT send any portion of
the original packet back to the client.  The server MUST disconnect the client
after sending the error message.

\item[\ttfamily source] \hfill \\
The source of this message.  Clients MAY use '*' to refer to themselves.
Servers MUST NOT use '*' in any message they transmit.

\item[\ttfamily dest] \hfill \\
The destination of this message.  Commands that do not require a destination
ignore this field and implementations SHOULD use '*'.

\item[\ttfamily command] \hfill \\
The command this message is representing.

\item[\ttfamily (keys/values)] \hfill \\
Any parameters of the command must be encoded in key/value format, as follows:


\texttt{
\begin{tabular}{|l|l|l|l|l|}
  %\hline
  key & value ( & key & value \ldots )
  %\hline
\end{tabular}
}

% +-------------------------------------------------------------+
% | key  | terminator (\0)  | value  | terminator (\0)  | [...] |
% +-------------------------------------------------------------+

Keys and values are separated by \nul.  Keys may be specified more than once,
when they have multiple values.  There is no protocol support for including
multiple values in a single stanza; an implementation MUST send multiple values
by specifying the key multiple times with a single value each.

If a key has no value, the value MUST be set to '*'.
\end{description}



\section{Naming standards}
\sectionrule
For \texttt{source}, \texttt{dest}, \texttt{target}, and other fields that
require the name of an entity, the following naming standards apply:

\begin{description}
\item[Groups] \hfill \\
A group always begins with the octothorpe character, or \#.

\item[Servers] \hfill \\
A server always begins with the equal sign character, or =.

\item[Foreigns] \hfill \\
A 'foreign' is a user or group from a remote federated network, and always
begins with the ampersand character, or \&.

\item[Rosters] \hfill \\
A roster always begins with the exclamation point character, or !.

\item[Users] \hfill \\
A user is any name that does not begin with \#, =, \&, or !.

\item[Other reserved characters] \hfill \\
The * ? and , symbols are reserved.  They may not appear anywhere in any name.

\end{description}

Clients MUST NOT attempt to register or sign in with a handle beginning with the
reserved characters.  Servers MUST refuse to register or sign in a client
attempting to use a handle beginning with the reserved characters.



\section{Compression}
\sectionrule

All streams, excepting option negotiation on client connection, MUST be
compressed.  The server MUST support at least zlib compression.  The server MAY
support other compression methods and expose them in the \texttt{options} key
upon client connection.  Clients MUST only use compression methods supported by
the server, which servers provide in the \texttt{options} key upon connection.



\section{Encryption}
\sectionrule

All streams MUST be encrypted via the Transport Layer Security Protocol (TLS).
\cite{TLS}
