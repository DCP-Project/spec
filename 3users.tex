\chapter{Users}
\label{chap:users}

This chapter describes what a user is, how client connections are established
and handled, user and client management, and other related features of DCP.

\newpage



\section{What is a user?}
\sectionrule

In DCP, a \textbf{user} is a unique entity that connects to DCP and uses it for
communication.  This means living persons and automated scripts that provide
information on request.

Unlike protocols such as IRC\cite{rfc1459}, which provide for a special class of
user for "network services" (such as ChanServ/NickServ/OperServ), DCP requires
all management services to be part of a server, via the core protocol or via
extensions\ref{chap:extensions}.  Therefore, there are no "pseudo-users" that
have special powers or extend the server.

Also, unlike protocols such as IRC or XMPP\cite{xmpp}, which provide for a
special class of user for management of a network, typically called "operators"
or "opers", DCP does not make such a distinction; management of every facet of
the network, and the global internetwork, is handled by any user that has the
respective permission granted to them.  This facilitates easier management of
the operator roster and finer-grained controls as desired by a group, network,
or the internetwork.



\section{What is a client?}
\sectionrule

A \textbf{client} is a device that a user uses to connect to DCP.  A user may
have any number of clients connected at the same time; DCP treats every client
as the same.  Unlike XMPP's resources, clients are all treated equal; they all
receive every message sent to the user.  However, clients may have separate
metadata and settings.  Servers MUST relay one client's message to all other
connected clients for the user, so that these clients have context for replies.

A user may blacklist a client, by either its client name property, or its client
software property.  Clients matching this blacklist will not be allowed to
connect to DCP as that user.



\section{Connection management}
\sectionrule

When a client wishes to connect to DCP, it sends a \texttt{SIGNON} message, as
specified:

\begin{tabular}{l|l|l|l}
  \hline
  Key & Format (Req'd) & Description & Example \\
  \hline
  \hline
  \texttt{handle}       & String (Y) & The user's ID.           & Pwyll@foxkit.us \\
  \texttt{password}     & String (Y) & The user's password.     & w3lsh1e \\
  \texttt{client-name}  & String (Y) & The name of this client. & laptop \\
  \texttt{client-ver}   & String (Y) & The version of this client. & Minnow 1.0.7 Unix \\
  \texttt{options}      & CSV    (N) & Options.                 & \\
  \hline
\end{tabular}

When a server receives a \texttt{SIGNON} message, it MUST perform the following
operations:

\begin{enumerate}
  \item{Ensure the connecting IP is not on the global DCP blacklist.}
  \item{Ensure the connecting IP is not on the network's blacklist.}
  \item{Ensure the connecting IP is not on the server's blacklist.}
  \item{Prevent flooding by ensuring a limited number of connections from this
  IP in a specified time frame.  Server and network operators MUST be able to
  configure both the number of connections and the time frame.}
  \item{Ensure the handle is not on the network's blacklist.}
  \item{Ensure the handle is not on the server's blacklist.}
  \item{Ensure the client name (\texttt{client-name}) and client version
  (\texttt{client-ver}) are not blacklisted by the user.}
  \item{Ensure the handle and password authenticate correctly.}
\end{enumerate}

User authentication is implementation-specific and is not covered by this
specification.
